\documentclass[CJKutf8,utf8,slidestop,mathserif,hyperref={pdfpagemode=FullScreen}]{beamer}
\usepackage{CJKutf8}
\usepackage[utf8]{inputenc}  % char encoding
\setbeamercolor{titlelike}{parent=structure,bg=lightgray}
\setbeamercovered{transparent}
\usefonttheme{structurebold}
\usetheme[options]{CambridgeUS}
\usecolortheme{rose}
\useinnertheme[shadow]{rounded}
\usecolortheme{beaver}
\useoutertheme{infolines}

\title{墨明棋妙(试用t2t-beamer)}
\author{红衣小猫}
\date{2008年12月}
\institute[江山如画]{神州奇侠}\begin{document}
\begin{CJK*}{UTF8}{li}

\frame{\titlepage}

\frame{\tableofcontents}
\clearpage

\section{倾尽天下}
\begin{frame}
\frametitle{倾尽天下}
\end{frame}

\subsection{曲辞乐}
\begin{frame}
\frametitle{\strut}
\framesubtitle{曲辞乐}
倾尽天下

曲:河图

辞:Finale

乐:河图/湛泸昭雪
\end{frame}

\subsection{影视梗概}
\begin{frame}
\frametitle{\strut}
\framesubtitle{影视梗概}
周帝白炎死在称帝十载后的一个雪夜。

这个草莽出身的皇帝不喜奢华,

逼宫夺位后便废弃了前朝敬帝所建的华美宫室,

而每夜宿在帝宫内的九龙塔,

死时亦盘膝在塔顶石室几案前的蒲团上,

正对着壁上一幅画像。

倘有历过前朝的宫女在,定会认出,

那画上颜色无双的女子,

正是前朝敬帝所封的最后一位贵妃。
\end{frame}

\subsection{影视梗概}
\begin{frame}
\frametitle{\strut}
\framesubtitle{影视梗概}
原来在倾国的十年之后,

白炎终究追随那人而去。

他身后并未留下只言片语。

于是所有关于周朝开国皇帝的谜团,

都与那悬于九重宝塔之上、

隐在七重纱幕背后的画像,

一并被掩埋进厚重的史书里。

◎墨明棋妙原创音乐团队出品◎
\end{frame}

\subsection{刀戟声}
\begin{frame}
\frametitle{\strut}
\framesubtitle{刀戟声}
刀戟声共丝竹沙哑

谁带你看城外厮杀

七重纱衣 血溅了白纱

兵临城下六军不发

谁知再见已是 生死无话

当时缠过红线千匝

一念之差为人作嫁

那道伤疤 谁的旧伤疤

还能不动声色饮茶

踏碎这一场 盛世烟花
\end{frame}

\subsection{血染江山}
\begin{frame}
\frametitle{\strut}
\framesubtitle{血染江山}
血染江山的画

怎敌你眉间 一点朱砂

覆了天下也罢

始终不过 一场繁华

碧血染就桃花

只想再见 你泪如雨下

听刀剑喑哑

高楼奄奄一息 倾塌
\end{frame}

\subsection{命犯桃花}
\begin{frame}
\frametitle{\strut}
\framesubtitle{命犯桃花}
是说一生命犯桃花

谁为你算的那一卦

最是无瑕 风流不假

画楼西畔反弹琵琶

暖风处处 谁心猿意马

色授魂与颠倒容华

兀自不肯相对照蜡

说爱折花 不爱青梅竹马

到头来算的那一卦

终是为你 覆了天下
\end{frame}

\subsection{明月照亮天涯}
\begin{frame}
\frametitle{\strut}
\framesubtitle{明月照亮天涯}
明月照亮天涯

最后谁又 得到了蒹葭

江山嘶鸣战马

怀抱中那 寂静的喧哗

风过天地肃杀

容华谢后 君临天下

登上九重宝塔

看一夜 流星飒沓
\end{frame}

\subsection{岁月无声}
\begin{frame}
\frametitle{\strut}
\framesubtitle{岁月无声}
回到那一刹那

岁月无声也让人害怕

枯藤长出枝桠

原来时光已翩然轻擦

梦中楼上月下

站着眉目依旧的你啊

拂去衣上雪花

并肩看 天地浩大
\end{frame}

\subsection{岁月无声}
\begin{frame}
\frametitle{\strut}
\framesubtitle{岁月无声}
回到那一刹那

岁月无声也让人害怕

枯藤长出枝桠

原来时光已翩然轻擦

梦中楼上月下

站着眉目依旧的你啊

拂去衣上雪花

并肩看 天地浩大



梦中楼上月下

站着眉目依旧的你啊

拂去衣上雪花

并肩看 天地浩大
\end{frame}

\section{京云初见雨}
\begin{frame}
\frametitle{京云初见雨}
\end{frame}

\subsection{曲辞乐原创}
\begin{frame}
\frametitle{\strut}
\framesubtitle{曲辞乐原创}
曲:心然/Finale

辞乐:湛泸昭雪/Finale

原创:临安初雨-云荒只如初见
\end{frame}

\subsection{一生一世念}
\begin{frame}
\frametitle{\strut}
\framesubtitle{一生一世念}
倾我一生一世念

来如飞花散似烟

梦萦云荒第几篇

风沙滚滚去天边

醉里不知年华限

当时月下舞连翩

又见海上花如雪

几轮春光葬枯颜
\end{frame}

\subsection{清风不解语}
\begin{frame}
\frametitle{\strut}
\framesubtitle{清风不解语}
清风不解语

翻开发黄书卷

梦中身 朝生暮死一夕恋

一样花开一千年

独看沧海化桑田

一笑望穿一千年

几回知君到人间

千载相逢如初见
\end{frame}

\subsection{韶华调}
\begin{frame}
\frametitle{\strut}
\framesubtitle{韶华调}
韶华调

烟云逐涛

对琴酌

月影斑驳

何事愁

聚散忍眸

听离舟

朝夕回首……
\end{frame}

\subsection{临安初雨 一夜落红}
\begin{frame}
\frametitle{\strut}
\framesubtitle{临安初雨 一夜落红}
临安初雨 一夜落红

春水凝碧 断雁越澄空

挥袖抚琴 七弦玲珑

芦苇客舟 雨朦胧

那年竹楼惘然如梦

纤指红尘 醉影笑惊鸿

皓月长歌 把酒临风

倾杯畅饮尽长虹
\end{frame}

\subsection{浮云事 尊前休说}
\begin{frame}
\frametitle{\strut}
\framesubtitle{浮云事 尊前休说}
浮云事 尊前休说

弹指间 昨日堪留

韶华易逝 岂料星移半昼

蓦回首 舟过群山万重


何处江湖何处留
\end{frame}

% LaTeX2e with latex-beamer class code generated by txt2tags 2.3.2 (http://txt2tags.sf.net)
% cmdline: txt2tags test-ltb.t2t
\end{CJK*}
\end{document}
